%%%%%%%%%%%%%%%%%%%%%%%%%%%%%%%%%%%%%%%%%%%%%%%%%%%%%%%%%%%%%%%%%%%%
%%%           Vorlage für eine Ausarbeitung an der DHBW          %%%
%%%                                                              %%%
%%%      Bereiche die bearbeitet werden müssen werden durch      %%%
%%%      einen solchen Kommentarblock eingeleitet und enden      %%%
%%%      mit der nächsten Trennlinie.                            %%%
%%%                                                              %%%
%%%      In dieser Datei müssen folgende Bereiche bearbeitet     %%%
%%%      werden:                                                 %%%
%%%      - Angaben zur Arbeit                                    %%%
%%%      - EIGENE KAPITEL EINFÜGEN                               %%%
%%%                                                              %%%
%%%      Benötigte Seiten und Verzeichnisse können unter         %%%
%%%      "Einführung und Verzeichnisse" ein- bzw. auskommentiert %%%
%%%      werden.                                                 %%%
%%%                                                              %%%
%%%%%%%%%%%%%%%%%%%%%%%%%%%%%%%%%%%%%%%%%%%%%%%%%%%%%%%%%%%%%%%%%%%%

\documentclass[a4paper,12pt]{article}
\usepackage[left=2.5cm,right=2.5cm,top=2.5cm,bottom=2.5cm,includehead]{geometry}      % Einstellungen der Seitenränder
\usepackage[english, ngerman]{babel}                                                  % deutsche Silbentrennung
\usepackage[utf8]{inputenc}                                                           % Umlaute
\usepackage[T1]{fontenc}													                                    % Umlaute auch richtig ausgeben
\usepackage{newtxtext,newtxmath}                                                      % Font = Times New Roman
\usepackage{hyperref}
\usepackage[nottoc]{tocbibind}
\usepackage{fancyhdr}
\usepackage{setspace}
\usepackage[backend=bibtex, citestyle=authoryear, bibstyle=authoryear]{biblatex}      % Bibliothek für Zitate
\usepackage{csquotes}                                                                 % Zusatzpacket für Zitate
\usepackage{amsmath}                                                                  % Zurücksetzen der Tabellen- und Abbildungsnummerierung je Sektion
\usepackage[labelfont=bf,aboveskip=1mm]{caption}                                      % Bild- und Tabellenunterschrift (fett)
\usepackage[bottom,multiple,hang,marginal]{footmisc}                                  % Fußnoten [Ausrichtung unten, Trennung durch Seperator bei mehreren Fußnoten]
\usepackage{graphicx}  
\graphicspath{{./images/}}                                                            % Grafiken
\usepackage[dvipsnames]{xcolor}                                                       % Farbige Buchstaben
\usepackage{wrapfig}                                                                  % Bilder in Text integrieren
\usepackage{enumitem}                                                                 % Befehl setlist (Zeilenabstand für itemize Umgebung auf 1 setzen)
\usepackage{listings}                                                                 % Quelltexte
\definecolor{commentgreen}{RGB}{87,166,74}                                            % Kommentar-Farbe für Quellcode
\lstset{numbers=left, numberstyle=\tiny, numbersep=8pt, frame=single, framexleftmargin=15pt, breaklines=true, commentstyle=\color{commentgreen}}
\usepackage{tabularx}                                                                 % Tabellen
\usepackage{multirow}                                                                 % Mehrzeilige Tabelleneinträge
\usepackage[addtotoc]{abstract}                                                       % Abstract
\usepackage[nohyperlinks, printonlyused, withpage]{acronym}                           % Abkürzungen
\usepackage{dirtree}                                                                  % Ordnerstruktur (z.B. für Anhang)

%%%%%%%%%%%%%%%%%%%%%%%%%%%%%%%%%%%%%%%%%%%%%%%%%%%%%%%%%%%%%%%%%%%%
%%%                      Angaben zur Arbeit                      %%%
%%%%%%%%%%%%%%%%%%%%%%%%%%%%%%%%%%%%%%%%%%%%%%%%%%%%%%%%%%%%%%%%%%%%
\def\vFirmenlogoPfad{images/Bosch_DE.png}                  %% relativer Pfad Bsp.: images/Firmenlogo.png
\def\vDHBWLogoPfad{images/DHBW_logo.jpg}                          %% relativer Pfad Bsp.: images/DHBW_logo.jpg
\def\vUnterschrift{}                    %% Pfad zu Bild mit Unterschrift (für digitale Abgabe) Bsp.: images/Unterschrift.png

\def\vTitel{Analyse von Audiosignalen unter der Verwendung von Linear Predictive Coding}                           %% 
\def\vUntertitel{}                      %% 
\def\vArbeitstyp{Projektarbeit}                      %% Projektarbeit/Seminararbeit/Bachelorarbeit
\def\vArbeitsbezeichnung{T3000}              %% T1000/T2000/T3000

\def\vAutor{Henry Schuler}                           %% Vorname Nachname
\def\vMatrikelnummer{5220542}                  %% 7-stellige Zahl
\def\vKursKuerzel{TIT20}                     %% Bsp.: TIT20
\def\vPhasenbezeichnung{Praxisphase}               %% Praxisphase/Theoriephase
\def\vStudienJahr{dritte}                     %% erste/zweite/dritte
\def\vDHBWStandort{Ravensburg}                    %% Bsp.: Ravensburg
\def\vDHBWCampus{Friedrichshafen}                      %% Bsp.: Friedrichshafen
\def\vFakultaet{Technik}                       %% Technik/Wirtschaft
\def\vStudiengang{Informationstechnik}                     %% Informationstechnik/...

\def\vBetrieb{Robert Bosch GmbH}                         %% 
\def\vBearbeitungsort{Blaichach}                 %% 
\def\vAbteilung{TEF-EAT31}                       %% 
\def\vBetreuer{Maximilian Main}                        %% Vorname Nachname

\def\vAbgabedatum{\today}               %% DD. MONTH YYYY
\def\vBearbeitungszeitraum{02.01.2023 - 17.04.2023}            %% DD.MM.YYYY - DD.MM.YYYY


%%%%%%%%%%%%%%%%%%%%%%%%% Eigene Kommandos %%%%%%%%%%%%%%%%%%%%%%%%%
% Definition von \gqq{}: Text in Anführungszeichen
\newcommand{\gqq}[1]{\glqq #1\grqq}
% Spezielle Hervorhebung von Schlüsselwörtern
\newcommand{\textOrdner}[1]{\texttt{#1}}
\newcommand{\textVariable}[1]{\texttt{#1}}
\newcommand{\textKlasse}[1]{\texttt{#1}}
\newcommand{\textFunktion}[1]{\texttt{#1}}


%%%%%%%%%%%%%%%%%%%% Zitatbibliothek einbinden %%%%%%%%%%%%%%%%%%%%%
\addbibresource{./literatur/literatur.bib}


%%%%%%%%%%%%%%%%%%%%%%%% PDF-Einstellungen %%%%%%%%%%%%%%%%%%%%%%%%%
\hypersetup{
  bookmarksopen=false,
	bookmarksnumbered=true,
	bookmarksopenlevel=0,
  pdftitle=\vTitel,
  pdfsubject=\vTitel,
  pdfauthor=\vAutor,
  pdfborder={0 0 0},
	pdfstartview=Fit,
  pdfpagelayout=SinglePage
}


%%%%%%%%%%%%%%%%%%%%%%%% Kopf- und Fußzeile %%%%%%%%%%%%%%%%%%%%%%%%
\pagestyle{fancy}
\setlength{\headheight}{15pt}
\fancyhf{}
\fancyhead[R]{\thepage}


%%%%%%%%%%%%%%%%%%%%%%%%%%%%%% Layout %%%%%%%%%%%%%%%%%%%%%%%%%%%%%%
\onehalfspacing
\setlist{noitemsep}

\addto\captionsngerman{
  \renewcommand{\figurename}{Abb.}
  \renewcommand{\tablename}{Tab.}
}
\numberwithin{table}{section}                               % Tabellennummerierung je Sektion zurücksetzen
\numberwithin{figure}{section}                              % Abbildungsnummerierung je Sektion zurücksetzen
\renewcommand{\thetable}{\arabic{section}.\arabic{table}}   % Tabellennummerierung mit Section
\renewcommand{\thefigure}{\arabic{section}.\arabic{figure}} % Abbildungsnummerierung mit Section
\renewcommand{\thefootnote}{\arabic{footnote}}              % Sektionsbezeichnung von Fußnoten entfernen

\renewcommand{\multfootsep}{, }                             % Mehrere Fußnoten durch ", " trennen


%%%%%%%%%%%%%%%%%%%%%%%%%%%%% Dokument %%%%%%%%%%%%%%%%%%%%%%%%%%%%%

\begin{document}


  %%%%%%%%%%%%%%%%%%% Einführung und Verzeichnisse %%%%%%%%%%%%%%%%%%%
  \pagenumbering{Roman}

  \begin{titlepage}
  \begin{minipage}{6in}
    \vspace*{-2cm}
    \centering
    \hspace{-2cm}
	\ifx\vFirmenlogoPfad\empty
	\else
    \raisebox{-0.5\height}{\includegraphics[height=4cm]{\vFirmenlogoPfad}}
  \fi
	\hfill
	\ifx\vDHBWLogoPfad\empty
	\else
   	\raisebox{-0.5\height}{\includegraphics[height=4cm]{\vDHBWLogoPfad}}
	\fi
  \end{minipage}
  \begin{center}
    \vspace*{0.5cm}
    \Huge\textbf{\vTitel}\\
		\ifx\vUntertitel\empty
		\else
			\Large\rm\vUntertitel\\
		\fi
		\vspace*{2cm}
		\Large\textbf{\vArbeitstyp}
		\ifx\vArbeitsbezeichnung\empty
		\else
			\textbf{\vArbeitsbezeichnung}
		\fi
		\\
		\normalsize
		über die \vPhasenbezeichnung\ des \vStudienJahr{n}\ Studienjahrs \\
		\vspace*{1cm}
		an der Fakultät für \vFakultaet\\
		im Studiengang \vStudiengang\\
		\vspace*{0.5cm}
		an der DHBW \vDHBWStandort\\
		\ifx\vDHBWCampus\empty
		\else
		Campus \vDHBWCampus\\
		\fi
		\vspace*{0.5cm}
		von\\
		\ifx\vAutor\empty
		\else
			\vAutor\\
		\fi
		\vspace*{1cm}
		\vAbgabedatum
		\vfill
  \end{center}
  \begin{tabular}{ll}
    Bearbeitungszeitraum:&\vBearbeitungszeitraum\\
    Matrikelnummer, Kurs:&\vMatrikelnummer, \vKursKuerzel\\
	  Dualer Partner:&\vBetrieb\\
	  Betreuer des Dualen Partners:&\vBetreuer\\
  \end{tabular}
\end{titlepage}
\newpage
\setcounter{page}{2}
  % \thispagestyle{empty}
\section*{\Huge{Sperrvermerk}}

\addcontentsline{toc}{section}{Sperrvermerk}
gemäß Ziffer 1.1.13 der Anlage 1 zu §§ 3, 4 und 5  der Studien- und Prüfungsordnung für die Bachelorstudiengänge im Studienbereich Technik der Dualen Hochschule Baden-Würt­tem­berg vom 29.09.2017.\\

\noindent \gqq{Der Inhalt dieser Arbeit darf weder als Ganzes noch in Auszügen Personen außerhalb des Prüfungsprozesses und des Evaluationsverfahrens zugänglich gemacht werden, sofern keine anders lautende Genehmigung vom Dualen Partner vorliegt.}

\vfill
\leavevmode
\newline
\parbox{6cm}{\strut\centering \vBearbeitungsort, \vAbgabedatum\hrule\strut\centering\footnotesize Ort, Datum} 
\hfill
\ifx\vUnterschrift\empty
\parbox{6cm}{\strut\hspace{1pt} \vAbteilung\hrule\strut\centering\footnotesize Abteilung, Unterschrift}
\else
\parbox{6cm}{\strut\hspace{1pt} \vAbteilung, \parbox[b]{3cm}{\vspace{-10cm}\includegraphics[width=3cm]{\vUnterschrift}}\hrule\strut\centering\footnotesize Abteilung, Unterschrift}
\fi
\vspace{1cm}

\newpage
  \thispagestyle{empty}
\section*{\Huge{Selbstständigkeitserklärung}}

\addcontentsline{toc}{section}{Selbstständigkeitserklärung}
gemäß Ziffer 1.1.13 der Anlage 1 zu §§ 3, 4 und 5  der Studien- und Prüfungsordnung für die Bachelorstudiengänge im Studienbereich Technik der Dualen Hochschule Baden-Würt­tem­berg vom 29.09.2017.

\noindent Ich versichere hiermit, dass ich meine Bachelorarbeit (bzw. Projektarbeit oder Studienarbeit bzw. Hausarbeit) mit dem Thema: 
\begin{center}
	\Large\textbf{\vTitel}
\end{center}
selbstständig verfasst und keine anderen als die angegebenen Quellen und Hilfsmittel benutzt habe. Ich versichere zudem, dass die eingereichte elektronische Fassung mit der gedruckten Fassung übereinstimmt.

\vfill
\leavevmode
\newline
\parbox{6cm}{\strut\centering \vBearbeitungsort, \vAbgabedatum\hrule\strut\centering\footnotesize Ort, Datum} 
\hfill
\ifx\vUnterschrift\empty
\parbox{6cm}{\strut\hspace{1pt} \vAbteilung\hrule\strut\centering\footnotesize Abteilung, Unterschrift}
\else
\parbox{6cm}{\strut\hspace{1pt} \vAbteilung, \parbox[b]{3cm}{\vspace{-10cm}\includegraphics[width=3cm]{\vUnterschrift}}\hrule\strut\centering\footnotesize Abteilung, Unterschrift}
\fi
\vspace{1cm}

\newpage
  \phantomsection
\newenvironment{keywords}{
	\begin{flushleft}
	\small	
	\textbf{
		\iflanguage{ngerman}{Schlüsselwörter}{\iflanguage{english}{Keywords}{}}
	}
}{\end{flushleft}}

% Deutsche Zusammenfassung
\begin{abstract}
	
\end{abstract}

% Schlüsselwörter Deutsch
\begin{keywords}
	
\end{keywords}


\selectlanguage{english}
% Englisches Abstract
\begin{abstract}

\end{abstract}

% Schlüsselwörter Englisch
\begin{keywords}

\end{keywords}


\selectlanguage{ngerman}
\newpage
  \tableofcontents
\newpage
  \section*{Abkürzungsverzeichnis}
\addcontentsline{toc}{section}{Abkürzungsverzeichnis}
\begin{acronym}
  \acro{DHBW}[DHBW]{Duale Hochschule Ba\-den-\-Würt\-tem\-berg}
  \acroplural{DHBW}[DHBW]{Dualen Hochschule Ba\-den-\-Würt\-tem\-berg}
  \acro{FFT}[FFT]{Fast Fourier Transform}
  \acro{LPC}[LPC]{Linear Predicitve Coding}
  \acro{LPCC}[LPCC]{Linear Prediction Cepstral Coefficient}
  \acro{MFCC}[MFCC]{Mel-frequency Cepstral Coefficients}
  \acro{AR}[AR]{Autoregression}
  \acro{NN}[NN]{Neuronales Netz}
  \acroplural{NN}[NN]{Neuronalen Netzes}
\end{acronym}
\newpage
  \listoffigures
\newpage
  \listoftables
\newpage
  \lstlistoflistings
\addcontentsline{toc}{section}{Listings}
\newpage
  % \section*{Vorwort}
\addcontentsline{toc}{section}{Vorwort}
\newpage


  %%%%%%%%%%%%%%%%%%%%%%%%%%%%% Kapitel %%%%%%%%%%%%%%%%%%%%%%%%%%%%%%
  \pagestyle{fancy}
  \fancyhead[L]{\nouppercase{\rightmark}}    % Abschnittsname im Header
  \pagenumbering{arabic}

  %%%%%%%%%%%%%%%%%%%%%%%%%%%%%%%%%%%%%%%%%%%%%%%%%%%%%%%%%%%%%%%%%%%%
  %%%%                   EIGENE KAPITEL EINFÜGEN                  %%%%
  %%%%%%%%%%%%%%%%%%%%%%%%%%%%%%%%%%%%%%%%%%%%%%%%%%%%%%%%%%%%%%%%%%%%
  \section{Einleitung}\label{sec:Einleitung}
Im Rahmen des Informatikstudiums an der \acp{DHBW} Ravensburg muss im dritten Studienjahr eine Studienarbeit abgelegt werden.
Die Hochschule stellt dafür eine Auswahl an Themen zur Verfügung.
Eines dieser Themen beschäftigt sich mit der Problematik der Sprecherauthentifizierung, wobei es Nutzern ermöglicht werden soll, sich über ihre Stimme zu authentifizieren.
In dieser Arbeit sollen die Grundlagen für die Bearbeitung dieser Studienarbeit behandelt werden.

\subsection{Kontext}
Damit ein Zusammenhang zwischen Stimme und Audioaufzeichnung hergestellt werden kann, müssen stimm\-spezifische Merkmale aus dem aufgezeichneten Stimmsignal extrahiert werden.
Im Bereich der Sprecherauthentifizierung haben sich zwei Verfahren zur Berechnung stimm\-spezifischer Merkmale etabliert: \ac{MFCC} und \ac{LPC} \autocite[vgl.][S. 116]{sidorov_text-independent_2010}\autocite[vgl.][S. 726]{chelali_text_2017}.
Während mittels des \ac{MFCC} Verfahrens versucht wird, die Funktionsweise des menschlichen Ohrs abzubilden, versucht das \ac{LPC} Verfahren die Eigenschaften des menschlichen Vokaltrakts aus dem Audiosignal zu extrahieren \autocite[vgl.][S. 117]{sidorov_text-independent_2010}.
Die erhaltenen Werte können anschließend für das Training eines \acp{NN} verwendet werden, welches die Klassifizierung neuer Datensätze während des Authentifizierungsprozesses übernimmt.

\subsection{Ziel der Arbeit}
Im Rahmen dieser Arbeit soll das \ac{LPC} Verfahren genauer untersucht werden.
Dazu soll ein Programm erstellt werden, welches ein gegebenes Audiosignal mittels \ac{LPC} in eine vordefinierte Anzahl an Koeffizienten umwandelt.
In einem weiteren Schritt soll der Zusammenhang zwischen den berechneten Koeffizienten und der sprechenden Person unter Verwendung eines vereinfachten \acp{NN} aufgezeigt werden.

\subsection{Vorgehensweise}
% 2. Grundlagen: Signalvorverarbeitung + LPC berechnung
% 3. Technische Umsetzung: Applikation
% 4. Validierung: NN Aufbau + Ergebnis.
% 5. Kritische Reflexion und Ausblick
Die Arbeit unterteilt sich in fünf Kapitel.
Im Anschluss an die Einleitung stellt Kapitel~\ref{sec:Grundlagen} die für diese Arbeit relevanten Grundlagen vor.
Kapitel~\ref{sec:TechnischeUmsetzung} kombiniert die vorgestellten Verfahren zu einem ausführbaren Programm.
Die Ergebnisse des erstellten Programms werden in Kapitel~\ref{sec:Validierung} validiert.
Abschließend werden die Erkenntnisse in Kapitel~\ref{sec:Ausblick} interpretiert und die Arbeit wird mit einem Ausblick abgeschlossen.

  %%%%%%%%%%%%%%%%%%%%%%% Literaturverzeichnis %%%%%%%%%%%%%%%%%%%%%%%
  \phantomsection
\addcontentsline{toc}{section}{Literatur}
\printbibliography
\newpage


  %%%%%%%%%%%%%%%%%%%%%%%%%%%%%% Anhang %%%%%%%%%%%%%%%%%%%%%%%%%%%%%%
  \renewcommand{\thetable}{\Alph{section}.\arabic{table}}
  \renewcommand{\thefigure}{\Alph{section}.\arabic{figure}}
  \renewcommand{\thelstlisting}{\Alph{section}.\arabic{lstlisting}}
  \pagenumbering{Alph}

  \begin{appendix}
  \section{Anhang}
  \subsection{AudioPreprocessor}
  \lstset{escapeinside={\#(*@}{@*)}}
  \lstinputlisting[caption={Signalvorverarbeitung}, captionpos=b, language=Python, label=code:AudioPreprocessor]{code/preprocessing/AudioPreprocessor.py}
\end{appendix}
\end{document}