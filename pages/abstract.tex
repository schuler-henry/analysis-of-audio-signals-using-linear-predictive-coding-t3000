\phantomsection
\newenvironment{keywords}{
	\begin{flushleft}
	\small	
	\textbf{
		\iflanguage{ngerman}{Schlüsselwörter}{\iflanguage{english}{Keywords}{}}
	}
}{\end{flushleft}}

% Deutsche Zusammenfassung
\begin{abstract}
	Um Sprecher anhand von Stimmaufzeichnungen in einem System zu authentifizieren, muss das analoge Audiosignal in digitale Parameter umgewandelt werden, die einen Bezug zu der sprechenden Person ermöglichen.
	Diese Arbeit beschäftigt sich mit der Berechnung von Linear Predictive Coefficients, welche Eigenschaften der Stimmerzeugung im Vokaltrakt modellieren.
	Der Zusammenhang zwischen Sprecher und berechneten Koeffizienten wird unter der Verwendung eines Neuronalen Netzes überprüft.
	In der Auswertung mit einem kleinen Datensatz von 10 Personen zeigt sich eine Vorhersagegenauigkeit von 70,54 Prozent, wodurch der grundsätzliche Zusammenhang gezeigt ist. 
\end{abstract}

% Schlüsselwörter Deutsch
\begin{keywords}
	Linear Predictive Coding, Sprecherauthentifizierung, Framing, Fensterfunktion
\end{keywords}


\selectlanguage{english}
% Englisches Abstract
\begin{abstract}
	To authenticate speakers via recordings of their voices, the analog audio recording has to be converted into digital parameters related to the speaker.
	For this reason, this study deals with calculating linear predictive coefficients, which are used to model human voice production in the vocal tract.
	The connection between speakers and calculated coefficients is checked using a neural network.
	Using a small data set consisting of ten different speakers, a prediction accuracy of 70.54 percent is achieved.
	Therefore the connection between the speaker and the coefficients is proven. 
\end{abstract}

% Schlüsselwörter Englisch
\begin{keywords}
	Linear Predictive Coding, Speaker Authentication, Framing, Windowing
\end{keywords}


\selectlanguage{ngerman}
\newpage