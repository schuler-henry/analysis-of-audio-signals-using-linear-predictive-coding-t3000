\section{Einleitung}\label{sec:Einleitung}
Im Rahmen des Informatikstudiums an der \acp{DHBW} Ravensburg muss im dritten Studienjahr eine Studienarbeit abgelegt werden.
Die Hochschule stellt dafür eine Auswahl an Themen zur Verfügung.
Eines dieser Themen beschäftigt sich mit der Problematik der Sprecherauthentifizierung, wobei es Nutzern ermöglicht werden soll, sich über ihre Stimme zu authentifizieren.
In dieser Arbeit sollen die Grundlagen für die Bearbeitung dieser Studienarbeit behandelt werden.

\subsection{Kontext}
Damit ein Zusammenhang zwischen Stimme und Audioaufzeichnung hergestellt werden kann, müssen stimm\-spezifische Merkmale aus dem aufgezeichneten Stimmsignal extrahiert werden.
Im Bereich der Sprecherauthentifizierung haben sich zwei Verfahren zur Berechnung stimm\-spezifischer Merkmale etabliert: \ac{MFCC} und \ac{LPC} \autocite[vgl.][S. 116]{sidorov_text-independent_2010} \autocite[vgl.][S. 726]{chelali_text_2017}.
Während mittels des \ac{MFCC}-Verfahrens versucht wird, die Funktionsweise des menschlichen Ohrs abzubilden, versucht das \ac{LPC}-Verfahren die Eigenschaften des menschlichen Vokaltrakts aus dem Audiosignal zu extrahieren \autocite[vgl.][S. 117]{sidorov_text-independent_2010}.
Die erhaltenen Werte können anschließend für das Training eines \acp{NN} verwendet werden, welches die Klassifizierung neuer Datensätze während des Authentifizierungsprozesses übernimmt.

\subsection{Ziel der Arbeit}
Im Rahmen dieser Arbeit soll das \ac{LPC}-Verfahren genauer untersucht werden.
Dazu soll ein Programm erstellt werden, welches ein gegebenes Audiosignal mittels \ac{LPC} in eine vordefinierte Anzahl an Koeffizienten umwandelt.
In einem weiteren Schritt soll der Zusammenhang zwischen den berechneten Koeffizienten und der sprechenden Person unter Verwendung eines vereinfachten \acp{NN} aufgezeigt werden.

\subsection{Vorgehensweise}
% 2. Grundlagen: Signalvorverarbeitung + LPC berechnung
% 3. Technische Umsetzung: Applikation
% 4. Validierung: NN Aufbau + Ergebnis.
% 5. Kritische Reflexion und Ausblick
Die Arbeit unterteilt sich in fünf Kapitel.
Im Anschluss an die Einleitung stellt Kapitel~\ref{sec:Grundlagen} die für diese Arbeit relevanten Grundlagen vor.
Kapitel~\ref{sec:TechnischeUmsetzung} kombiniert die vorgestellten Verfahren zu einem ausführbaren Programm.
Die Ergebnisse des erstellten Programms werden in Kapitel~\ref{sec:Validierung} validiert.
Abschließend werden die Erkenntnisse in Kapitel~\ref{sec:Ausblick} interpretiert und die Arbeit wird mit einem Ausblick abgeschlossen.